 
\documentclass[DIN, pagenumber=false, fontsize=11pt, parskip=half]{scrartcl}

\usepackage{amsmath}
\usepackage{amsfonts}
\usepackage{amssymb}
\usepackage{enumitem}
\usepackage[utf8]{inputenc} % this is needed for umlauts
\usepackage[ngerman]{babel} % this is needed for umlauts
\usepackage[T1]{fontenc} 
\usepackage{commath}
\usepackage{xcolor}
\usepackage{booktabs}
\usepackage{float}
\usepackage{tikz-timing}
\usepackage{tikz}
\usepackage{multirow}
\usepackage{colortbl}
\usepackage{xstring}
\usepackage{circuitikz}
\usepackage{listings} % needed for the inclusion of source code
\usepackage[final]{pdfpages}
\usepackage{subcaption}

\usetikzlibrary{calc,shapes.multipart,chains,arrows}

\newcommand{\Prb}[1]{P(\text{#1})}
\newcommand{\CPr}[2]{P(\text{#1}|\text{#2})}
\DeclareMathOperator*{\argmax}{arg\,max}
\DeclareMathOperator*{\argmin}{arg\,min}

\title{Pattern Recognition}
\author{Tim Luchterhand, Paul Nykiel, Jonas Strauch (Group P)}

\begin{document}
    \maketitle
    \section{Decision Surfaces via Gaussian Density Functions [Pen and Paper]}
    \begin{table}[H]
        \centering
        \begin{tabular}{cccccc}
            \toprule
            $\mu_1$ & $\Sigma_1$ & $\mu_2$ & $\Sigma_2$ & Decision Surface & Slice \\
            \midrule
            $\begin{pmatrix} 0 \\ 0 \end{pmatrix}$ & $\begin{pmatrix} 1 & 0 \\ 0 & 1 \end{pmatrix}$ &
                $\begin{pmatrix} 2 \\ 0 \end{pmatrix}$ & $\begin{pmatrix} 3 & 0 \\ 0 & 3 \end{pmatrix}$ &
                    b & e  \\
            $\begin{pmatrix} 0 \\ 0 \end{pmatrix}$ & $\begin{pmatrix} 1 & 0 \\ 0 & 1 \end{pmatrix}$ &
                $\begin{pmatrix} 2 \\ 0 \end{pmatrix}$ & $\begin{pmatrix} 1 & 0 \\ 0 & 3 \end{pmatrix}$ &
                   c & f \\
            $\begin{pmatrix} 0 \\ 0 \end{pmatrix}$ & $\begin{pmatrix} 2 & 0 \\ 0 & 3 \end{pmatrix}$ &
                $\begin{pmatrix} 2 \\ 0 \end{pmatrix}$ & $\begin{pmatrix} 6 & 0 \\ 0 & 1 \end{pmatrix}$ &
                   a & d \\
            \bottomrule
        \end{tabular}
        \caption{Parameters for the Gaussian functions which define the decision surfaces.}
    \end{table}
    
    \paragraph{First Row}
    Both covariance matrices are multiples of the identity matrix, therefor the decision boundary is a circle. The circle is shifted to the left on the $x_1$-axis,
    because the gaussians are not centered.
\end{document}
