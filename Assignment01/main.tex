 
\documentclass[DIN, pagenumber=false, fontsize=11pt, parskip=half]{scrartcl}

\usepackage{amsmath}
\usepackage{amsfonts}
\usepackage{amssymb}
\usepackage{enumitem}
\usepackage[utf8]{inputenc} % this is needed for umlauts
\usepackage[ngerman]{babel} % this is needed for umlauts
\usepackage[T1]{fontenc} 
\usepackage{commath}
\usepackage{xcolor}
\usepackage{booktabs}
\usepackage{float}
\usepackage{tikz-timing}
\usepackage{tikz}
\usepackage{multirow}
\usepackage{colortbl}
\usepackage{xstring}
\usepackage{circuitikz}
\usepackage{listings} % needed for the inclusion of source code
\usepackage[final]{pdfpages}
\usepackage{subcaption}

\usetikzlibrary{calc,shapes.multipart,chains,arrows}

\newcommand{\Prb}[1]{P(\text{#1})}
\newcommand{\CPr}[2]{P(\text{#1}|\text{#2})}
\DeclareMathOperator*{\argmax}{arg\,max}
\DeclareMathOperator*{\argmin}{arg\,min}

\title{Pattern Recognition}
\author{Tim Luchterhand, Paul Nykiel, Jonas Strauch (Group P)}

\begin{document}
    \maketitle
    \section{Discrete Naive Bayer Classifier}
    \begin{enumerate}
        \item
        Length: Interval-scale, Sweetness: ordinal-scale, Colour: nominal-scale
        \item
        $\Prb{Banana} \approx \frac{600}{1900} = 31,58\%$
        \item
        $\Prb{Long} \approx \frac{550}{1900} = 28,95\%$
        \item
        $\CPr{Banana}{Long} = \frac{500}{500 + 50} = 90,91\%$
        \item
        $\CPr{Long}{Banana} = \frac{\CPr{Banana}{Long} \cdot \Prb{Long}}{\Prb{Banana}} = 83,33\%$
        \item
        \begin{align*}
            &\argmax_\omega{\left( P\left(\omega | (\text{Medium, Sweet, Green})^\text{T}\right) \right)} = \\
            &\argmax_\omega{\left( P(\text{Medium} | \omega) \cdot P(\text{Sweet} | \omega) \cdot P(\text{Green} | \omega) \cdot P(\omega) \right)} \\
        \end{align*}
        For $\omega = \text{Banana}$: $P(\omega | x) \cdot p(x) = 0$

        For $\omega = \text{Papaya}$: $P(\omega | x) \cdot p(x) = 0,0042$

        For $\omega = \text{Apple}$: $P(\omega | x) \cdot p(x) = 0,0453$

        $\Rightarrow$ The fruit is most likely an apple.
\end{enumerate}

    \section{Misclassification Costs}
    \begin{enumerate}
        \item
        Instead of maximizing a probability the costs of misclassification should be minimized. A misclassification occurs if e.g. a sample of class 1 is classified
        as an instance of class 2. The $l_2$ term for example can be interpreted as the probability of observing an instance of the other class ($\omega_1$) scaled by the cost of
        wrongly classifying a sample as $\omega_2$. Thus class $\omega_2$ is choosen if the probability of class $\omega_1$ or the cost of confusing a sample
        of $\omega_1$ with one of $\omega_2$ are low compared to $l_1$ and vice versa.
        \item
        \begin{enumerate}
            \item
            $\lambda_{12} = 1$
            \item
            $\lambda_{12} = 0.5$
            \item
            $\lambda_{12} = \frac{2}{3}$
        \end{enumerate}
        \item
        \begin{align*}
            &l_1(x) \stackrel{!}{=} l_2(x) \\
            &\Leftrightarrow \lambda_{21} \cdot p(x|\omega_2) \cdot P(\omega_2) = \lambda_{12} \cdot p(x|\omega_1) \cdot P(\omega_1) \\
            &\Leftrightarrow \lambda_{21} \cdot \frac{1}{\sqrt{\pi}} \cdot \exp{-(x-1)^2} \cdot (1-p) = \lambda_{12} \cdot \frac{1}{\sqrt{\pi}} \cdot \exp(-x^2) \cdot p \\
            &\Leftrightarrow \frac{\lambda_{12}}{\lambda_{21}} \frac{p}{1-p} = \exp(2x-1) \\
            &\Leftrightarrow \frac{1}{2} \left( \log\left( \frac{\lambda_{12}}{\lambda_{21}} \frac{p}{1-p} \right) + 1 \right) = x
        \end{align*}
        \item
        Plots
        \begin{figure}[H]
            \centering
            \begin{subfigure}[t]{0.49\textwidth}
                \centering
                \includegraphics[width=\textwidth]{A24_Likelihoods.eps}
            \end{subfigure}
            \begin{subfigure}[t]{0.49\textwidth}
                \centering
                \includegraphics[width=\textwidth]{A24_Losses.eps}
            \end{subfigure}
        \end{figure}
        \item
        Plots
        \begin{figure}[H]
            \centering
            \begin{subfigure}[t]{0.49\textwidth}
                \centering
                \includegraphics[width=\textwidth]{A24_Likelihoods.eps}
            \end{subfigure}
            \begin{subfigure}[t]{0.49\textwidth}
                \centering
                \includegraphics[width=\textwidth]{A25_Losses.eps}
            \end{subfigure}
        \end{figure}
        \item
        Plots
        \begin{figure}[H]
            \centering
            \begin{subfigure}[t]{0.49\textwidth}
                \centering
                \includegraphics[width=\textwidth]{A24_Likelihoods.eps}
            \end{subfigure}
            \begin{subfigure}[t]{0.49\textwidth}
                \centering
                \includegraphics[width=\textwidth]{A26_Losses.eps}
            \end{subfigure}
        \end{figure}
        \item
        \begin{align*}
            \widehat{x}_0(1, 1, p) \stackrel{!}{=} \frac{1}{2} \\
            \Rightarrow p = \frac{1}{2}
        \end{align*}
        \begin{figure}[H]
            \centering
            \begin{subfigure}[t]{0.49\textwidth}
                \centering
                \includegraphics[width=0.9\textwidth]{A27a_Losses.eps}
                \subcaption{}
            \end{subfigure}
            \begin{subfigure}[t]{0.49\textwidth}
                \centering
                \includegraphics[width=0.9\textwidth]{A27b_Losses.eps}
                \subcaption{}
            \end{subfigure}
            \begin{subfigure}[t]{0.49\textwidth}
                \centering
                \includegraphics[width=0.9\textwidth]{A27c_Losses.eps}
                \subcaption{}
            \end{subfigure}
        \end{figure}
    \end{enumerate}
\end{document}
