 
\documentclass[DIN, pagenumber=false, fontsize=11pt, parskip=half]{scrartcl}

\usepackage{amsmath}
\usepackage{amsfonts}
\usepackage{amssymb}
\usepackage{enumitem}
\usepackage[utf8]{inputenc} % this is needed for umlauts
\usepackage[ngerman]{babel} % this is needed for umlauts
\usepackage[T1]{fontenc} 
\usepackage{commath}
\usepackage{xcolor}
\usepackage{booktabs}
\usepackage{float}
\usepackage{tikz-timing}
\usepackage{tikz}
\usepackage{multirow}
\usepackage{colortbl}
\usepackage{xstring}
\usepackage{circuitikz}
\usepackage{listings} % needed for the inclusion of source code
\usepackage[final]{pdfpages}
\usepackage{subcaption}

\usetikzlibrary{calc,shapes.multipart,chains,arrows}

\newcommand{\Prb}[1]{P(\text{#1})}
\newcommand{\CPr}[2]{P(\text{#1}|\text{#2})}
\DeclareMathOperator*{\argmax}{arg\,max}
\DeclareMathOperator*{\argmin}{arg\,min}

\title{Pattern Recognition}
\author{Tim Luchterhand, Paul Nykiel, Jonas Strauch (Group P)}

\begin{document}
    \maketitle
    \section{Maximum Likelihood Parameter Estimation}
    \begin{enumerate}
        \item
            \begin{eqnarray*}
                L(p) &=& P({X_1 = n_1}) \land P({X_2 = n_2}) \land P({X_3 = n_3}) \\
                &\stackrel{\text{Variables are independent}}{=}& P({X_1 = n_1}) \cdot P({X_2 = n_2}) \cdot P({X_3 = n_3}) \\
                &\stackrel{\text{Same distribution}}{=}& p \cdot {(1-p)}^{n_1 - 1} \cdot p \cdot {(1-p)}^{n_2 - 1} \cdot p \cdot {(1-p)}^{n_3 - 1} \\
                &\stackrel{\text{Actual values}}{=}& p \cdot {(1-p)}^{1} \cdot p \cdot {(1-p)}^{0} \cdot p \cdot {(1-p)}^{4} \\
                &=& p^3 \cdot {(1-p)}^5
            \end{eqnarray*}
        \item
            First the derivative is calculated:
            \begin{equation*}
                L'(p) = \frac{\text{d}}{\text{d}p} L(p) = 3 p^2 {(1-p)}^5 - 5 p^3 {(1-p)}^4
            \end{equation*}
            Then an extrema is calculated:
            \begin{eqnarray*}
                L'(p) &\stackrel{!}{=}& 0 \\
                \Leftrightarrow 3 p^2 {(1-p)}^5 - 5 p^3 {(1-p)}^4 &=& 0 \\
                \Leftrightarrow 3 p^2 {(1-p)}^5 &=& 5 p^3 {(1-p)}^4 \\
                \stackrel{0 < p < 1}{\Leftrightarrow} 3 (1-p) &=& 5 p \\
                \Leftrightarrow 3 - 3p &=& 5p \\
                \Leftrightarrow 3 &=& 8p \\
                \Leftrightarrow \frac{3}{8} &=& p
            \end{eqnarray*}
        \item
            \lstinputlisting[language=Python]{A1.py}
            \begin{figure}[H]
                \includegraphics[width=\textwidth]{A1.eps}
                \caption{Output of the script}
            \end{figure}
    \end{enumerate}
\end{document}
